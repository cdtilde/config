% Created 2014-05-01 Thu 14:59
\documentclass[11pt]{article}
\usepackage[utf8]{inputenc}
\usepackage[T1]{fontenc}
\usepackage{fixltx2e}
\usepackage{graphicx}
\usepackage{longtable}
\usepackage{float}
\usepackage{wrapfig}
\usepackage{rotating}
\usepackage[normalem]{ulem}
\usepackage{amsmath}
\usepackage{textcomp}
\usepackage{marvosym}
\usepackage{wasysym}
\usepackage{amssymb}
\usepackage{hyperref}
\tolerance=1000
\usepackage{minted}
\author{Zachrisen}
\date{\today}
\title{Org-Mode Cheat Sheet}
\hypersetup{
  pdfkeywords={},
  pdfsubject={},
  pdfcreator={Emacs 24.4.50.1 (Org mode 8.2.6)}}
\begin{document}

\maketitle

\section{Org Documents}
\label{sec-1}

\begin{center}
\begin{tabular}{ll}
Key & Function\\
C-c C-x $\backslash$ & Show entities as UTF-8 characters (e.g., $\alpha$)\\
<s TAB & Insert \#+BEGIN$_{\text{SRC}}$ .. \#+END$_{\text{SRC}}$ template (more)\\
\end{tabular}
\end{center}


\subsection{Character Formatting}
\label{sec-1-1}

\index{Formatting}

\begin{itemize}
\item \texttt{*bold*} \textbf{example}
\item \texttt{/italic/} \emph{example}
\item \texttt{\_underlined\_} \uline{example}
\item \texttt{\textasciitilde{}code\textasciitilde{}} \verb~example~
\item \texttt{+strike-through+} \sout{example}
\end{itemize}

\index{Formatting!Code}
Here's an example of \verb~max(int a, int b)~ code in text.

\subsection{Paragraph Formatting}
\label{sec-1-2}

Some random text here, followed by a quote:

\begin{quote}
Everything should be made as simple as possible,
but not any simpler -- Albert Einstein
\end{quote}

The regular text continues here. 

You can insert a hard paragraph break\\
 with two \emph{$\backslash$}$\backslash$'s at the end of a line.

\subsection{Miscellaneous}
\label{sec-1-3}

\begin{itemize}
\item \# and a whitespace are comments
\item 5 dashes will be a horizontal rule
\item $\square$ A checkbox here and more details!
\item $\boxtimes$ A checked checkbox here
\item I have a footnote! \footnote{This is the text of the footnote!}
\end{itemize}

\subsection{Math}
\label{sec-1-4}
\begin{itemize}
\item Greek characters are written like this: \verb~\alpha~ $\alpha$ $\beta$ $\gamma$
\item Subscripts and superscripts are done with \verb~^~ and \verb~_~: x$^{\text{y}}$, $\sigma$$_{\text{var}}$
\end{itemize}

Latex equations can be written like this:

\begin{equation}
x=\sqrt{b}
\end{equation}

If $a^2=b$ and \( b=2 \), then the solution must be
either $$ a=+\sqrt{2} $$ or \[ a=-\sqrt{2} \]


\section{Code}
\label{sec-2}

\subsection{Keybindings}
\label{sec-2-1}
\begin{center}
\begin{tabular}{ll}
Key & Function\\
C-c ' & Edit current code block in separate window\\
\end{tabular}
\end{center}

\subsection{Including in Text}
\label{sec-2-2}
Source code is included and referenced like this:

\begin{verbatim}
#+BEGIN_SRC emacs-lisp -n -r
    (save-excursion                                                       (x)
       (goto-char (point-min)))                                           (y)
#+END_SRC
In line [[(x)]] we remember the current position.  [[(y)][Line (y)]]
jumps to point-min.
\end{verbatim}

\begin{minted}[linenos,firstnumber=1]{common-lisp}
(save-excursion
   (goto-char (point-min)))
\end{minted}

In line 1 we remember the current position. Line 2
jumps to point-min.

Options:
\begin{description}
\item[{\verb~-n~}] number the lines
\item[{\verb~+n~}] continue the numbering from the previous snippet
\item[{\verb~-r~}] don't display the labels in the source code
\end{description}

\subsection{Evaluating Code}
\label{sec-2-3}

This code snippet will print the code, and then the value returned by evaluating the code at export time.
\begin{minted}[linenos,firstnumber=1]{common-lisp}
(point-max)
\end{minted}

The results will be placed here:

\begin{verbatim}
6704
\end{verbatim}



\section{Tables}
\label{sec-3}
\subsection{Keybindings}
\label{sec-3-1}

\begin{center}
\begin{tabular}{ll}
Key & Function\\
\hline
C-c ' & edit all formulas\\
C-c - & insert a hline\\
C-c = & edit column formula\\
C-c ? & get info about current cell\\
C-c $\vert{}$ & convert region to table\\
C-c \^{} & sort the lines in the region\\
C-c C-c & realign table\\
C-u C-c = & edit field formula\\
C-u C-u C-c * & recalc table\\
M-left/right & move column left or right\\
M-s-down & move current row down and add a row above\\
M-s-left & delete current column\\
M-s-right & move current column right and insert a column to the left\\
M-up/down & move row up or down\\
\end{tabular}
\end{center}

\subsection{Miscellaneous}
\label{sec-3-2}
\begin{description}
\item[{M-x}] org-table-export
\item Types of formulas:
\begin{description}
\item[{:=}] field formula
\item[{=}] column formula
\end{description}
\end{description}



\subsection{Example Table}
\label{sec-3-3}

Code:

\begin{verbatim}
#+CONSTANTS: pi=3.14
#+CAPTION: Very important table with numbers in it
#+TBLNAME: summary-cost 
| Item           |   | Annual Cost | Monthly Cost |            |
|                |   |         <r> |          <r> |      <r10> |
|----------------+---+-------------+--------------+------------|
| /              | < |             |            > |            |
| Startup Cost   | 1 |         $10 |          $12 |     $1,232 |
| Something else | 2 |    $232,323 |          $24 |    $45,555 |
| Something      | 2 |      $3,231 |          $24 |         $0 |
| Final thing    | 3 |      $3,422 |          $36 |         $0 |
|----------------+---+-------------+--------------+------------|
| Totals         | 8 |    $238,986 |          $96 |         $0 |
#+TBLFM: $4=$2*12;C%d::@8$3=vsum(@I..II);C%d::$5=$5;C%d
\end{verbatim}

Output:


\begin{table}sidewaystable
\caption{\label{summary-cost}Very important table with numbers in it}
\centering
\begin{tabular}{l|rrr|r}
Item &  & Annual Cost & Monthly Cost & \\
\hline
Startup Cost & 1 & \$10 & \$12 & \$1,232\\
Something else & 2 & \$232,323 & \$24 & \$45,555\\
Something & 2 & \$3,231 & \$24 & \$0\\
Final thing & 3 & \$3,422 & \$36 & \$0\\
\hline
Totals & 8 & \$238,986 & \$96 & \$0\\
\end{tabular}
\end{table}


Notes:
\begin{itemize}
\item <r10> forces right-align and makes the column 10 characters
\item / in first column generates vertical column group separators
\item Referring to a value in the table, the total in column 3 is \texttt{\$238,986}. Code: \verb~src_emacs-lisp[:var d=summary-cost[9,2]]{d}~
\item Constants can be referred to as \verb~$pi~
\item C is a custom format specifier. It says that the input may have \$ and , in it, and to strip those out, perform the calculation, then add \$ and , back in before displaying the results.
\end{itemize}


\section{Tables of Content}
\label{sec-4}

\listoflistings


\listoftables

\subsection{And some more stuff here}
\label{sec-4-1}
I guess\ldots{}

\begin{center}
\begin{tabular}{lrrr}
Month & Days & Nr sold & per day\\
\hline
Jan & 23 & 55 & 2.4\\
Feb & 21 & 16 & 0.8\\
March & 22 & 278 & 12.6\\
\end{tabular}
\end{center}

\#+ BEGIN RECEIVE ORGTBL salesfigures
\#+ END RECEIVE ORGTBL salesfigures


\#+ BEGIN RECEIVE ORGTBL xyz
\#+ END RECEIVE ORGTBL xyz

\begin{center}
\begin{tabular}{rr}
A & B\\
22 & 23\\
12 & 66\\
\end{tabular}
\end{center}
% Emacs 24.4.50.1 (Org mode 8.2.6)
\end{document}
